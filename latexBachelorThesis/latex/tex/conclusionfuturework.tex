\section{Conclusion and Future Work}
\subsection{Conclusion}

The latent space of variational autoencoders for aerial images is investigated and visualized 
which allows for an increased understanding of the performed unsupervised clustering and the 
latent representation in general. For that purpose the thesis first offers a brief introduction
into neural networks and convolutional architecture as well as a comprehensive explanation 
of how VAEs improve standard autoencoders to generate a useful continuous latent space.\\ 

Experiments with different VAE architectures are conducted regarding their reconstruction quality where a fully
convolutional architecture yielded the most accurate reconstructions and this VAE's latent space is analyzed.
Using principle component analysis and t-stochastic neighbor embedding the dimension of the latent representations
is reduced and the results are visualized. It becomes clear that the VAE clusters in an unsupervised manner.
This means, up to the latent layer, it learned the features for which it clustered the inputs.\\

The formed clusters are compared with available height and classification data revealing that the learned clusters
correspond to different topographic classes to a reasonable degree which is not the case for the average height of
an image. 

While the VAE additionally groups the images by their average color it definitely also learns very complex features 
like types or directions of roads.\\

Thereby it is found that smaller latent dimensions lead to better visualizations of distinct clusters which suggests
that the smaller encoding size forces better clustering to be learned. However, it is also possible that the 
shorter latent representations are just easier to interpret and the visualization techniques just worked better.
That being said, a smaller latent space also means worse reconstructions and there is a tradeoff to be made between
an interpretable latent space on the one hand and reconstruction quality on the other hand.


\subsection{Future Work}

From the viewpoint of this thesis the next step in, in the regard of constructing a multi task taxonomy, is to apply
the techniques used for understanding the latent space of VAEs to different layers in single task models. This insight
about the latent information learned in the different layers of the single task model then opens possibilities
to systematically determine optimal multi task architectures instead of relying on ad hoc testing.\\

To more directly establish on this thesis it would be interesting to analyze the unsupervised clustering performed by
the VAEs for additional topographic classes if semantic labeling data for those classes was available.
Moreover, a normalization method for the classes should be employed since in the used dataset an aerial
image is more likely to predominantly have pixels labeled as ground instead of pixels labeled as building.
This normalization could take the overall occurrence of each class in the whole dataset into account and reveal
additional learned features in the latent layer.


